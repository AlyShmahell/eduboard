\documentclass{article}
\usepackage{xcolor}
\pagecolor[rgb]{0.0,0.0,0.15}
\color[rgb]{1,1,1}
\title{Deriving a Solution to Quadratic Residues}
\author{by : Aly Shmahell}
\date{Jan $ 3^{rd} $ 2016}
\begin{document}
\maketitle
\begin{flushleft}
\begin{center}
if 
\end{center}
$ x^{2} \equiv a \bmod p \Rightarrow a \equiv x^{2} \bmod p $\space\space\space\space....(1) \newline
\begin{center}
then :
\end{center}
 $ a^{\frac{(p-1)}{2}} \equiv x^{2^{\frac{(p-1)}{2}}} \equiv x^{p-1} \equiv 1 \bmod p$ \newline
\begin{center}
assuming 
\end{center}
 $ \frac{(p-1)}{2} = 2k+1 $\space\space\space\space....(2) \newline
\begin{center}
therefor :
\end{center}
 $ a^{2k+1} \equiv 1 \bmod p $ \space\space\space\space....(3) \newline
\begin{center}
And :
\end{center}
$ a^{2k+1} * a \equiv a^{2k+2} \bmod p $ \newline\newline
$ a \equiv a^{2k+2} \bmod p $\space\space\space\space .... according to (3) \newline\newline
$ x^{2} \equiv a^{2(k+1)} \bmod p $ .... according to (1) \newline\newline
$ x \equiv a^{k+1} \bmod p $\space\space\space\space .... where $ x $ can be calculated from (2)
\end{flushleft}
\end{document}